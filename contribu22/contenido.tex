% ex: ts=2 sw=2 sts=2 et filetype=tex
% SPDX-License-Identifier: CC-BY-SA-4.0

\section{¿Qué es Linux?}

\begin{frame}[c]{Historia}
  \begin{columns}
    \column{0.6\textwidth}
     \begin{itemize}
      \item Linux comienza en 1991 como un proyecto personal del estudiante
        finlandés \textbf{Linus Benedict Torvalds}$^{\hyperlink{creditos}{[1]}}$.
      \pausa
      \item Crea un nuevo núcleo de un sistema operativo libre, bajo la
        \textbf{GNU General Public License v2} (GPL v2)
      \pausa
      \item El núcleo Linux resultante ha estado marcado por un crecimiento
        constante a lo largo de su historia.
     \end{itemize}
    \column{0.4\textwidth}
      \pausa
      \begin{center}
        \includegraphics[scale=0.6]{lfs/linus.png}
      \end{center}
  \end{columns}
\end{frame}

\begin{frame}[fragile]
  \frametitle{El segundo anuncio en usenet$^{\hyperlink{creditos}{[2]}}$}
  \begin{block}{}
  \begin{verbatim}
Newsgroups: comp.os.minix
Subject: What would you like to see most in minix?
Date: 25 Aug 91 20:57:08 GMT

Hello everybody out there using minix -

I'm doing a (free) operating system (just a hobby, won't be big and
professional like gnu) for 386(486) AT clones.  This has been brewing
since april, and is starting to get ready.  I'd like any feedback on
things people like/dislike in minix, as my OS resembles it somewhat
(same physical layout of the file-system (due to practical reasons)
among other things).
  \end{verbatim}
  \end{block}
\end{frame}

\begin{frame}[c]{Historia de las versiones del kernel$^{\hyperlink{creditos}{[3]}}$}
  \begin{center}
    \includegraphics[scale=0.39]{lfs/linux_kernel_timeline.png}
  \end{center}
\end{frame}

\begin{frame}[c]{Versiones activas del kernel}
  Hay varias categorías principales en las que pueden caer los
  lanzamientos del kernel$^{\hyperlink{creditos}{[4]}}$:

  \vspace{\baselineskip}
  \begin{description}
    \item[Prepatch] Los kernels de preparche o "RC" (candidanto a
      liberación) son versiones preliminares del kernel de la línea principal
      (mainline) que están dirigidas principalmente a otros desarrolladores
      de kernel y entusiastas de Linux.

      \vspace{\baselineskip}
      Deben compilarse desde la fuente y, por lo general, contienen nuevas
      características que deben probarse antes de que puedan colocarse en
      una versión estable.

      \vspace{\baselineskip}
      Linus Torvalds mantiene y publica los kernels previos al parche. 
  \end{description}
\end{frame}

\begin{frame}[c]{Versiones activas del kernel}
  \begin{description}
    \item[Mainline] Linus Torvalds mantiene el árbol principal (mainline).
      Es el árbol donde se presentan todas las funciones nuevas y donde
      ocurre todo el desarrollo nuevo y emocionante.

      \vspace{\baselineskip}
      Los nuevos núcleos principales se lanzan cada 9 o 10 semanas. 
  \end{description}
\end{frame}

\begin{frame}[c]{Versiones activas del kernel}
  \begin{description}
    \item[Stable] Después de que se lanza cada núcleo principal (mainline),
      se considera "estable".

      \vspace{\baselineskip}
      Todas las correcciones de errores para un kernel estable se
      retroalimentan (backported) desde el árbol de la línea principal
      y las aplica un mantenedor de kernel estable designado.

      \vspace{\baselineskip}
      Por lo general, solo hay unas pocas versiones del kernel con corrección
      de errores hasta que el siguiente kernel de la línea principal esté
      disponible, a menos que se designe como un
      "kernel de mantenimiento a largo plazo".

      \vspace{\baselineskip}
      Las actualizaciones estables del kernel se publican según sea
      necesario, generalmente una vez a la semana. 
  \end{description}
\end{frame}

\begin{frame}[c]{Versiones activas del kernel}
  \begin{description}
    \item[Longterm] Por lo general, se proporcionan varias versiones del
      kernel de "\textbf{mantenimiento a largo plazo}" con el fin de
      respaldar las correcciones de errores para los árboles del kernel
      más antiguos.

      \vspace{\baselineskip}
      Solo se aplican correcciones de errores importantes a dichos núcleos
      y, por lo general, no se lanzan con mucha frecuencia,
      especialmente para los árboles más antiguos. 
  \end{description}
\end{frame}

\begin{frame}[c]{Núcleos de lanzamiento a largo plazo}
  \begin{table}[]
  \begin{tabular}{llll}
    \textbf{Versión} &  \textbf{Mantenedor} & \textbf{Liberado} & \textbf{Fin de vida} \\
     & & & \textbf{proyectado} \\
    \rowcolor{editorGray}
    5.15&Greg Kroah-Hartman \& Sasha Levin 	&2021-10-31 	&Oct, 2023 \\
    5.10&Greg Kroah-Hartman \& Sasha Levin 	&2020-12-13 	&Dec, 2026 \\
    \rowcolor{editorGray}
    5.4&Greg Kroah-Hartman \& Sasha Levin 	&2019-11-24 	&Dec, 2025 \\
    4.19&Greg Kroah-Hartman \& Sasha Levin 	&2018-10-22 	&Dec, 2024 \\
    \rowcolor{editorGray}
    4.14&Greg Kroah-Hartman \& Sasha Levin 	&2017-11-12 	&Jan, 2024 \\
    4.9&Greg Kroah-Hartman \& Sasha Levin 	&2016-12-11 	&Jan, 2023 \\
  \end{tabular}
  \end{table}
\end{frame}



\begin{frame}[c]{Historia de las versiones del kernel 4.x$^{\hyperlink{creditos}{[3]}}$}
  \begin{center}
    \includegraphics[scale=0.7]{lfs/linux_kernel_timeline_4.png}
  \end{center}
\end{frame}

\begin{frame}[c]{Historia de las versiones del kernel 5.x$^{\hyperlink{creditos}{[3]}}$}
  \begin{center}
    \includegraphics[scale=0.7]{lfs/linux_kernel_timeline_5.png}
  \end{center}
\end{frame}


%\section{}
%
%\begin{frame}[c]{}
%  \vspace{\baselineskip}
%\end{frame}
%
%\begin{frame}[fragile]
%  \frametitle{ufw - Cortafuegos sin complicaciones}
%  \begin{lstlisting}[language=Bash,numbers=none]
%  \end{lstlisting}
%\end{frame}
