% ex: ts=2 sw=2 sts=2 et filetype=tex
% SPDX-License-Identifier: CC-BY-SA-4.0

\section{¿Qué es Linux?}

\begin{frame}[c]{Antes de empezar}
  Descarga el código del kernel con:
  \begin{itemize}
    \item git clone git://git.kernel.org/pub/scm/linux/kernel/git/torvalds/linux.git
    \item git clone
      \href{https://git.kernel.org/pub/scm/linux/kernel/git/torvalds/linux.git}
      {https://git.kernel.org/pub/scm/linux/kernel/git/torvalds/linux.git}
  \end{itemize}

  \vspace{\baselineskip}
  otra manera de descargar es con:
  \begin{itemize}
    \item wget -c
      \href{https://git.kernel.org/pub/scm/linux/kernel/git/torvalds/linux.git/clone.bundle}
      {https://git.kernel.org/pub/scm/linux/kernel/git/torvalds/linux.git/clone.bundle}
    \item curl -LO
      \href{https://git.kernel.org/pub/scm/linux/kernel/git/torvalds/linux.git/clone.bundle}
      {https://git.kernel.org/pub/scm/linux/kernel/git/torvalds/linux.git/clone.bundle}
  \end{itemize}

  \vspace{\baselineskip}
  La información de los bundles esta en 
  \href{https://www.kernel.org/cloning-linux-from-a-bundle.html}
  {https://www.kernel.org/cloning-linux-from-a-bundle.html}
\end{frame}

\begin{frame}[c]{Linux}
  \begin{itemize}
    \item Linux es un núcleo mayormente libre semejante al núcleo de Unix.
    \pausa
    \item Linux es uno de los principales ejemplos de software libre y de
      código abierto.
    \pausa
    \item Linux está licenciado bajo la GPL v2 salvo el hecho que tiene
      blobs binarios no-libres.
    \pausa
    \item Está desarrollado por colaboradores de todo el mundo.
  \end{itemize}
\end{frame}

\begin{frame}[c]{Tux}
  \begin{columns}
    \column{0.3\textwidth}
      \begin{center}
        \includegraphics[scale=0.1]{tux.png}
      \end{center}
    \column{0.7\textwidth}
      \begin{itemize}
        \item Tux es el nombre de la mascota oficial de Linux.
        \pause
        \item Creado por Larry Ewing en 1996
        \pause
        \item Es un pequeño pingüino de aspecto risueño y cómico basado
          en una imagen que Linus Torvalds encontró en un servidor FTP.
      \end{itemize}
  \end{columns}
\end{frame}

\begin{frame}[c]{¿Cómo contribuir?}
  \begin{itemize}
    \item Su modelo es por correo electrónico
    \pausa
    \item Los parches tienes que estar bien documentados
    \pausa
    \item Linux \emph{mainline} es la rama principal
  \end{itemize}
\end{frame}
