% ex: ts=2 sw=2 sts=2 et filetype=tex
% SPDX-License-Identifier: CC-BY-SA-4.0

\begin{frame}[c]{Antes de empezar}
  Descarga el código del kernel con:
  \begin{itemize}
    \item git clone git://git.kernel.org/pub/scm/linux/kernel/git/torvalds/linux.git
    \item git clone
      \href{https://git.kernel.org/pub/scm/linux/kernel/git/torvalds/linux.git}
      {https://git.kernel.org/pub/scm/linux/kernel/git/torvalds/linux.git}
  \end{itemize}

  \vspace{\baselineskip}
  otra manera de descargar es usando bundles$^{\hyperlink{creditos}{[6]}}$:
  \begin{itemize}
    \item wget -c
      \href{https://git.kernel.org/pub/scm/linux/kernel/git/torvalds/linux.git/clone.bundle}
      {https://git.kernel.org/pub/scm/linux/kernel/git/torvalds/linux.git/clone.bundle}
    \item curl -LO
      \href{https://git.kernel.org/pub/scm/linux/kernel/git/torvalds/linux.git/clone.bundle}
      {https://git.kernel.org/pub/scm/linux/kernel/git/torvalds/linux.git/clone.bundle}
  \end{itemize}
\end{frame}

\section{¿Qué es Linux?}

\begin{frame}[c]{Historia}
  \begin{columns}
    \column{0.4\textwidth}
      \begin{center}
        \includegraphics[scale=0.6]{lfs/linus.png}
      \end{center}
    \column{0.6\textwidth}
     \begin{itemize}
      \item Linux comienza en 1991 como un proyecto personal del estudiante
        finlandés \textbf{Linus Benedict Torvalds}$^{\hyperlink{creditos}{[1]}}$.
      \pausa
      \item Crea un nuevo núcleo de un sistema operativo libre, bajo la
        \textbf{GNU General Public License v2} (GPL v2)
      \pausa
      \item El núcleo Linux resultante ha estado marcado por un crecimiento
        constante a lo largo de su historia.
     \end{itemize}
  \end{columns}
\end{frame}

\begin{frame}[fragile]
  \frametitle{El segundo anuncio en usenet$^{\hyperlink{creditos}{[2]}}$}
  \begin{block}{}
  \begin{verbatim}
Newsgroups: comp.os.minix
Subject: What would you like to see most in minix?
Date: 25 Aug 91 20:57:08 GMT

Hello everybody out there using minix -

I'm doing a (free) operating system (just a hobby, won't be big and
professional like gnu) for 386(486) AT clones.  This has been brewing
since april, and is starting to get ready.  I'd like any feedback on
things people like/dislike in minix, as my OS resembles it somewhat
(same physical layout of the file-system (due to practical reasons)
among other things).
  \end{verbatim}
  \end{block}
\end{frame}

\begin{frame}[c]{Historia de las versiones del kernel$^{\hyperlink{creditos}{[3]}}$}
  \begin{center}
    \includegraphics[scale=0.39]{lfs/linux_kernel_timeline.png}
  \end{center}
\end{frame}

\begin{frame}[c]{Versiones activas del kernel}
  Hay varias categorías principales en las que pueden caer los
  lanzamientos del kernel$^{\hyperlink{creditos}{[4]}}$:

  \vspace{\baselineskip}
  \begin{description}
    \item[Prepatch] Los kernels de preparche o "RC" (candidanto a
      liberación) son versiones preliminares del kernel de la línea principal
      (mainline) que están dirigidas principalmente a otros desarrolladores
      de kernel y entusiastas de Linux.

      \vspace{\baselineskip}
      Deben compilarse desde la fuente y, por lo general, contienen nuevas
      características que deben probarse antes de que puedan colocarse en
      una versión estable.

      \vspace{\baselineskip}
      Linus Torvalds mantiene y publica los kernels previos al parche.
  \end{description}
\end{frame}

\begin{frame}[c]{Versiones activas del kernel}
  \begin{description}
    \item[Mainline] Linus Torvalds mantiene el árbol principal (mainline).
      Es el árbol donde se presentan todas las funciones nuevas y donde
      ocurre todo el desarrollo nuevo y emocionante.

      \vspace{\baselineskip}
      Los nuevos núcleos principales se lanzan cada 9 o 10 semanas.
  \end{description}
\end{frame}

\begin{frame}[c]{Versiones activas del kernel}
  \begin{description}
    \item[Stable] Después de que se lanza cada núcleo principal (mainline),
      se considera "estable".

      \vspace{\baselineskip}
      Todas las correcciones de errores para un kernel estable se
      retroalimentan (backported) desde el árbol de la línea principal
      y las aplica un mantenedor de kernel estable designado.

      \vspace{\baselineskip}
      Por lo general, solo hay unas pocas versiones del kernel con corrección
      de errores hasta que el siguiente kernel de la línea principal esté
      disponible, a menos que se designe como un
      "kernel de mantenimiento a largo plazo".

      \vspace{\baselineskip}
      Las actualizaciones estables del kernel se publican según sea
      necesario, generalmente una vez a la semana.
  \end{description}
\end{frame}

\begin{frame}[c]{Versiones activas del kernel}
  \begin{description}
    \item[Longterm] Por lo general, se proporcionan varias versiones del
      kernel de "\textbf{mantenimiento a largo plazo}" con el fin de
      respaldar las correcciones de errores para los árboles del kernel
      más antiguos.

      \vspace{\baselineskip}
      Solo se aplican correcciones de errores importantes a dichos núcleos
      y, por lo general, no se lanzan con mucha frecuencia,
      especialmente para los árboles más antiguos.
  \end{description}
\end{frame}

\begin{frame}[c]{Núcleos de lanzamiento a largo plazo}
  \begin{table}[]
  \begin{tabular}{llll}
    \textbf{Versión} &  \textbf{Mantenedor} & \textbf{Liberado} & \textbf{Fin de vida} \\
     & & & \textbf{proyectado} \\
    \rowcolor{editorGray}
    5.15&Greg Kroah-Hartman \& Sasha Levin 	&2021-10-31 	&Oct, 2023 \\
    5.10&Greg Kroah-Hartman \& Sasha Levin 	&2020-12-13 	&Dec, 2026 \\
    \rowcolor{editorGray}
    5.4&Greg Kroah-Hartman \& Sasha Levin 	&2019-11-24 	&Dec, 2025 \\
    4.19&Greg Kroah-Hartman \& Sasha Levin 	&2018-10-22 	&Dec, 2024 \\
    \rowcolor{editorGray}
    4.14&Greg Kroah-Hartman \& Sasha Levin 	&2017-11-12 	&Jan, 2024 \\
    4.9&Greg Kroah-Hartman \& Sasha Levin 	&2016-12-11 	&Jan, 2023 \\
  \end{tabular}
  \end{table}
\end{frame}

\begin{frame}[c]{Historia de las versiones del kernel 4.x$^{\hyperlink{creditos}{[3]}}$}
  \begin{center}
    \includegraphics[scale=0.7]{lfs/linux_kernel_timeline_4.png}
  \end{center}
\end{frame}

\begin{frame}[c]{Historia de las versiones del kernel 5.x$^{\hyperlink{creditos}{[3]}}$}
  \begin{center}
    \includegraphics[scale=0.7]{lfs/linux_kernel_timeline_5.png}
  \end{center}
\end{frame}

\section{Bienvenida}

\begin{frame}[c]{Trabajando con la comunidad de desarrollo del kernel}
  \begin{columns}
    \column{0.6\textwidth}
      Entonces, ¿quieres ser un desarrollador del kernel de
      Linux?$^{\hyperlink{creditos}{[5]}}$

      \vspace{\baselineskip}
      \begin{center}
        ¡Bienvenidos!
      \end{center}

      \vspace{\baselineskip}
      Si bien hay mucho que aprender sobre el kernel en un sentido técnico,
      también es importante aprender cómo funciona esta comunidad.
    \column{0.4\textwidth}
      \begin{center}
        \includegraphics[scale=0.5]{tux.png}
      \end{center}
  \end{columns}
\end{frame}

\section{Como hacer el desarrollo del kernel de Linux}

\begin{frame}[c]{Introducción}
  Entonces, ¿quieres aprender a convertirte en un desarrollador del kernel
  de Linux? O su gerente le ha dicho: "Escriba un controlador de Linux para
  este dispositivo"$^{\hyperlink{creditos}{[7]}}$.

  \vspace{\baselineskip}
  El kernel está escrito \textbf{principalmente en C}, con algunas partes
  dependientes de la arquitectura escritas en \textbf{ensamblador}.
  Se requiere una \underline{buena
  comprensión de C} para el desarrollo del kernel. No se requiere
  el lenguaje ensamblador (de cualquier arquitectura) a menos que planee
  hacer un desarrollo de bajo nivel para esa arquitectura.

\end{frame}

\begin{frame}[c]{Introducción}
  Aunque no son un buen sustituto de una sólida educación C y/o años de
  experiencia, los siguientes libros son buenos, en todo caso,
  como referencia$^{\hyperlink{creditos}{[7]}}$:

  \vspace{\baselineskip}
  \begin{itemize}
    \item “El lenguaje de programación C” de Kernighan y Ritchie [Prentice Hall]
    \item “Programación práctica en C” por Steve Oualline [O'Reilly]
    \item “C: Manual de referencia” de Harbison y Steele [Prentice Hall]
  \end{itemize}
\end{frame}

\begin{frame}[c]{Introducción}
  El kernel está escrito usando GNU C y la cadena de herramientas GNU.
  Si bien se adhiere al estándar \textbf{ISO
  C89}\footnote{Información desactualizada: oportunidad para contribuir},
  utiliza una serie de extensiones
  que no se incluyen en el estándar$^{\hyperlink{creditos}{[7]}}$.

  \vspace{\baselineskip}
  El kernel es un \textbf{entorno C independiente}, que \textbf{no}
  depende de la biblioteca C estándar,
  por lo que algunas partes del estándar C \underline{no son compatibles}.

  \vspace{\baselineskip}
  No se permiten divisiones largas largas arbitrarias ni puntos flotantes.
  A veces puede ser difícil comprender las suposiciones que tiene el núcleo
  sobre la cadena de herramientas y las extensiones que utiliza y,
  lamentablemente, no existe una referencia definitiva para ellas.
\end{frame}

\begin{frame}[c]{Introducción}
  Recuerde que está tratando de aprender a \textbf{trabajar con la comunidad de
  desarrollo existente}. Es un grupo diverso de personas, con
  \textbf{altos estándares}
  de codificación, estilo y procedimiento$^{\hyperlink{creditos}{[7]}}$.

  \vspace{\baselineskip}
  Estos estándares se han creado a lo largo del tiempo en función de lo que
  han descubierto que funciona mejor para un equipo tan grande y
  geográficamente disperso. Trate de \textbf{aprender lo más posible}
  sobre estos estándares con anticipación, ya que están bien documentados;
  \underline{no espere} que la gente se adapte a su forma de hacer
  las cosas o la de su empresa.
\end{frame}

\begin{frame}[c]{Asuntos legales}
  El código fuente del kernel de Linux se publica bajo \textbf{licencia GPL}.
  Consulte el archivo COPYING en el directorio principal del árbol de
  fuentes$^{\hyperlink{creditos}{[7]}}$.

  \vspace{\baselineskip}
  Las reglas de licencia del kernel de Linux y cómo usar los identificadores
  \textbf{SPDX} en el código fuente se describen en
  \href{https://git.kernel.org/pub/scm/linux/kernel/git/torvalds/linux.git/tree/Documentation/process/license-rules.rst}
  {Documentation/process/license-rules.rst}.

  \vspace{\baselineskip}
  Si tiene más preguntas sobre la licencia, comuníquese con un abogado y no
  pregunte en la lista de correo del kernel de Linux. Las personas en las
  listas de correo no son abogados y no debe confiar en sus declaraciones
  sobre asuntos legales.

  \vspace{\baselineskip}
  Para preguntas y respuestas comunes sobre la GPL, consulte:

  \vspace{\baselineskip}
  \href{https://www.gnu.org/licenses/gpl-faq.html}
  {https://www.gnu.org/licenses/gpl-faq.html}
\end{frame}

\begin{frame}[c]{Documentación}
  El árbol de fuentes del kernel de Linux tiene una gran variedad de
  \textbf{documentos que son invaluables para aprender a interactuar} con la
  comunidad del kernel$^{\hyperlink{creditos}{[7]}}$.

  \vspace{\baselineskip}
  Cuando se agregan nuevas funciones al kernel, se recomienda que también
  se agreguen \textbf{nuevos archivos de documentación} que expliquen cómo
  usar la función.

  \vspace{\baselineskip}
  Cuando un cambio en el kernel hace que cambie la interfaz que el kernel
  expone al espacio de usuario, se recomienda que envíe la información o
  un parche a las páginas del manual que explique el cambio al mantenedor
  de las páginas del manual \textbf{mtk.manpages@gmail.com},
  y copie (CC) la lista
  \textbf{linux-api@vger.kernel.org}.
\end{frame}

\begin{frame}[c]{Convertirse en un desarrollador de kernel}
  Si no sabe nada sobre el desarrollo del kernel de Linux, debe consultar
  el proyecto \textbf{Linux KernelNewbies}$^{\hyperlink{creditos}{[7]}}$:

  \begin{block}{}
    \begin{center}
      \href{https://kernelnewbies.org}{https://kernelnewbies.org}
    \end{center}
  \end{block}

  \vspace{\baselineskip}
  Consiste en una lista de correo útil donde puede hacer casi cualquier
  tipo de pregunta básica sobre el desarrollo del kernel
  (asegúrese de buscar primero en los archivos, antes de preguntar algo que
  ya se haya respondido en el pasado).

  \vspace{\baselineskip}
  También tiene un canal \textbf{IRC} que puede
  utilizar para hacer preguntas en tiempo real y una gran cantidad de
  documentación útil que es útil para aprender sobre el desarrollo del
  kernel de Linux.

\end{frame}

\begin{frame}[c]{Convertirse en un desarrollador de kernel}
  El sitio web tiene \textbf{información básica} sobre la organización del
  código, los subsistemas y los proyectos actuales (tanto dentro como fuera del
  árbol). También describe información logística básica, como cómo compilar
  un núcleo y aplicar un parche$^{\hyperlink{creditos}{[7]}}$.

  \vspace{\baselineskip}
  Si no sabe por dónde quiere empezar, pero quiere buscar alguna tarea
  para empezar a hacer para unirse a la comunidad de desarrollo del kernel,
  vaya al proyecto Linux Kernel Janitor:

  \begin{exampleblock}{}
    \begin{center}
      \href{https://kernelnewbies.org/KernelJanitors}
        {https://kernelnewbies.org/KernelJanitors}
    \end{center}
  \end{exampleblock}

  Es un gran lugar para comenzar. Describe una lista de problemas
  \textbf{relativamente simples} que deben limpiarse y corregirse dentro
  del árbol de fuentes del kernel de Linux. Trabajando con los
  desarrolladores a cargo de este proyecto, aprenderá los conceptos
  básicos para colocar su parche en el árbol del kernel de Linux, y
  posiblemente se le indicará en qué dirección trabajar a continuación,
  si aún no tiene una idea.
\end{frame}

\begin{frame}[c]{Convertirse en un desarrollador de kernel}
  Antes de realizar modificaciones reales al código del kernel de Linux,
  es imperativo comprender cómo funciona el código en cuestión.

  \vspace{\baselineskip}
  Para este propósito, nada mejor que \textbf{leerlo directamente}
  (la mayoría de los pedacitos están bien comentados), quizás incluso con
  la ayuda de herramientas especializadas.

  \vspace{\baselineskip}
  Una de esas herramientas que se recomienda especialmente es el proyecto
  \textbf{Linux Cross-Reference}, que puede presentar el código fuente en
  un formato de página web indexado y autorreferencial. Puede encontrar
  un excelente repositorio actualizado del código del kernel en:

  \begin{exampleblock}{}
    \begin{center}
      \href{https://elixir.bootlin.com/}{https://elixir.bootlin.com/}
    \end{center}
  \end{exampleblock}
\end{frame}

\begin{frame}[c]{El proceso de desarrollo}
  El proceso de desarrollo del kernel de Linux actualmente consta de
  algunas "\textbf{ramas}" diferentes del kernel principal y muchas
  ramas del kernel específicas de subsistemas diferentes.
  Estas diferentes ramas son$^{\hyperlink{creditos}{[7]}}$:

  \vspace{\baselineskip}
  \begin{itemize}
    \item Árbol principal (\textbf{mainline}) de Linus
    \item Varios árboles estables con múltiples números principales
    \item Árboles específicos del \href{https://git.kernel.org/}{subsistema}
    \item Árbol de pruebas de integración \textbf{linux-next}
  \end{itemize}
\end{frame}

\begin{frame}[c]{Informes de problemas}

  \begin{itemize}
    \item ¿Se enfrenta a una regresión con un \textbf{kernel vainilla}
      de la misma serie de la rama estable o de largo plazo?
    \item ¿Todavía es mantenida?
  \end{itemize}

  \vspace{\baselineskip}
  Entonces busque en \href{https://lore.kernel.org/lkml/}{LKML} y en
  los archivos de la \href{lista de correo
  estable}{https://lore.kernel.org/stable/} de Linux para encontrar
  informes coincidentes para unirse. Si no encuentra ninguno, instale
  la \href{https://kernel.org/}{última versión de esa
  serie}$^{\hyperlink{creditos}{[8]}}$.

  \vspace{\baselineskip}
  Si aún muestra el problema, repórtelo a la lista de correo estable
  (\textbf{stable@vger.kernel.org}) y copie (CC)la lista de regresiones
  (\textbf{regressions@lists.linux.dev}); idealmente, también copie (CC)
  el mantenedor y la lista de correo para el subsistema en cuestión.
\end{frame}

\begin{frame}[c]{Informes de problemas}
  En todos los demás casos, intente averiguar qué parte del kernel podría
  estar causando el problema. Consulte el archivo
  \href{https://git.kernel.org/pub/scm/linux/kernel/git/torvalds/linux.git/tree/MAINTAINERS}
  {MAINTAINERS} para saber cómo esperan que los desarrolladores
  sean informados sobre los problemas, que la mayoría de las veces será
  por correo electrónico con una lista de correo en copia
  (CC)$^{\hyperlink{creditos}{[8]}}$.

  \vspace{\baselineskip}
  Verifique los archivos del destino para buscar informes coincidentes;
  busque en \href{https://lore.kernel.org/lkml/}{LKML} y en la web también.
  Si no encuentra ninguno para unirse, instale el kernel principal más
  reciente. Si el problema está presente allí, envíe un informe.

\end{frame}

\begin{frame}[c]{Informes de problemas}
  El problema se solucionó allí, pero ¿le gustaría verlo resuelto en una
  serie estable o a largo plazo aún compatible?.

  \vspace{\baselineskip}
  Entonces instale su última versión. Si muestra el problema, busque el
  cambio que lo arregló en la línea principal y verifique si el
  \textbf{backporting} está en proceso o fue descartado; si no es
  ninguno, pregunte a quienes manejaron el
  cambio$^{\hyperlink{creditos}{[8]}}$.

  \vspace{\baselineskip}
  \begin{alertblock}{Observaciones generales}
    Al instalar y probar un kernel como se describe anteriormente,
    asegúrese de que sea estándar (\textbf{vainilla}) (sin parches y sin
    módulos complementarios).

    \vspace{\baselineskip}
    También asegúrese de que esté construido y funcionando en un
    entorno saludable y que no esté contaminado antes de que ocurra
    el problema.
  \end{alertblock}

\end{frame}

%\section{}
%
%\begin{frame}[c]{}
%  \vspace{\baselineskip}
%\end{frame}
%
%\begin{frame}[fragile]
%  \frametitle{ufw - Cortafuegos sin complicaciones}
%  \begin{lstlisting}[language=Bash,numbers=none]
%  \end{lstlisting}
%\end{frame}
